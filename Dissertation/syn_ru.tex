\counterwithout{figure}{chapter}
\counterwithout{table}{chapter}
\renewcommand{\figurename}{Рисунок}
\renewcommand{\tablename}{Таблица}
\renewcommand\thesubfigure{\asbuk{subfigure}}
\chapter*{Реферат}
\addcontentsline{toc}{chapter}{Реферат} %%To manually add unnumbered section, use the command \addcontentsline.

\begin{center}
    Общая характеристика диссертации
\end{center}

\paragraph*{Актуальность.}

\paragraph*{Цель исследования.}
\paragraph*{Научные задачи.}

\paragraph*{Методы исследования.}

\paragraph*{Основные положения, выносимые на защиту.}
%\begin{enumerate}
%    \item a%\statementOneRU
%    \item b%\statementTwoRU 
%\end{enumerate}

\paragraph*{Научная новизна.}

\paragraph*{Теоретическая значимость.}
\paragraph*{Практическая значимость.}
\paragraph*{Достоверность.}
\paragraph*{Аппробация работы.}
Основные результаты диссертации докладывались на следующих конференциях:
\begin{enumerate}
	\item \textcolor{blue}{The 22nd IFAC World Congress}, Yokohama, Japan.  9–-14 июля 2023 г.
	\item \textcolor{blue}{The 63rd IEEE Conference on Decision and Control (CDC-2024)}, Milan, Italy. 16--19 декабря 2024 г.
	\item \textcolor{blue}{XIV Всероссийское совещание по проблемам управления (ВСПУ-2024)}, Россия, Москва, ИПУ РАН. 17--20 июня 2024 г.
\end{enumerate}
\paragraph*{Личный вклад автора.}


\paragraph*{Объём и структура работы.}
Диссертация состоит из введения,
\formbytotal{totalchapter}{глав}{ы}{}{},
заключения и
\formbytotal{totalappendix}{приложен}{ия}{ий}{}.
%% на случай ошибок оставляю исходный кусок на месте, закомментированным
%Полный объём диссертации составляет  \ref*{TotPages}~страницу
%с~\totalfigures{}~рисунками и~\totaltables{}~таблицами. Список литературы
%содержит \total{citenum}~наименований.
%
Полный объём диссертации составляет
\formbytotal{TotPages}{страниц}{у}{ы}{}, включая
\formbytotal{totalcount@figure}{рисун}{ок}{ка}{ков} и
\formbytotal{totalcount@table}{таблиц}{у}{ы}{}.
Список литературы содержит
\formbytotal{citenum}{наименован}{ие}{ия}{ий}.


\newpage
\section*{Основное содержание работы}

В Главе~\ref{ch:ch1}...
%\begin{figure}
%	\centering
%	\includegraphics[width=0.4\linewidth]{images/knuth}
%	\caption{Knuth}
%	\label{fig:my_label2}
%\end{figure}
%\begin{table}
%	\centering
%	\captionsetup{justification=centering} % выравнивание подписи по-центру
%	\caption{Основные величины СИ}%\label{tab:unit:base}
%	\begin{tabular}{llc}
%		\toprule
%		Название  & Команда                 & Символ         \\
%		\midrule
%		Ампер     & \verb|\ampere| & \si{\ampere}   \\
%		Кандела   & \verb|\candela| & \si{\candela}  \\
%		\bottomrule
%	\end{tabular}
%\end{table}
%\begin{figure}
%	\captionsetup[subfloat]{position=below}
%	\centering
%	\subfloat[caption fig (a).]{\includegraphics[scale=0.5]{example-image-a}\label{fig2:sub1}} 
%	\hspace{0.5cm} %\qquad %\vfill
%	\subfloat[caption fig (b).]{\includegraphics[scale=0.5]{example-image-b}\label{fig2:sub2}}
%	\vspace{0.5cm}
%	\caption{Example using subfloat}\label{fig:subfloat}
%\end{figure}

\section*{Публикации автора по теме диссертации}

Основные результаты по теме диссертации изложены в \theAllPapers~публикациях. 
Из них \theScopusPapers~опубликовано в изданиях, индексируемых в базе цитирования Scopus, 
из них \theVakPapers~изданы в журналах, рекомендованных ВАК. 
%%Также имеется 1 свидетельство о государственной регистрации программ для ЭВМ.
%

В международных изданиях, индексируемых в базе данных Scopus, Web of Science:
\insertpapperScopus

%В изданиях из перечня ВАК РФ:\\

В иных изданиях:
\insertpapperOther
