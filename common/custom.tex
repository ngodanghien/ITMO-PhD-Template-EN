% author: hiennd, March 03 2024
% For english version.
\newtheorem{theorem}{Theorem}[chapter]
\newtheorem{assumption}{Assumption}[chapter] % Предположение or Допущение 1.1
\newtheorem{lemma}{Lemma}[chapter]
\newtheorem{proposal}{Proposition}[chapter]
\newtheorem{proposition}{Proposition}[chapter] % maybe
\newtheorem{corollary}{Corollary}[chapter]
\newtheorem{statement}{Statement}[chapter]
\newtheorem{remark}{Remark}[chapter]
\newtheorem{definition}{Definition}[chapter]
\newtheorem{example}{Example}[chapter]
\newtheorem{problem}{Problem}[chapter]
%\newtheorem{algorithm}{Алгоритм}[chapter]
%\newtheorem{iteration}{Итерационная схема}

% for russian, follow as VSPUart stye (VSPU-2024)
%Synopse: \newtheorem{name}[numbered_like]{title}
% default = plain : italic text, extra space above and below
\theoremstyle{definition}% definition : upright text, extra space above and below;

%\newtheorem{theorem-syn-ru}{Теорема}
\newtheorem{assumption-syn-ru}{Допущение} % Предположение or Допущение 1
\newtheorem{assumption-syn-en}{Assumption}
\newtheorem{lemma-syn-ru}{Лемма}
\newtheorem{lemma-syn-en}{Lemma}
\newtheorem{proposition-syn-ru}{Предложение} % maybe
\newtheorem{proposition-syn-en}{Proposition}
\newtheorem{corollary-syn-ru}{Следствие}
\newtheorem{corollary-syn-en}{Corollary}
\newtheorem{statement-syn-ru}{Утверждение}
\newtheorem{statement-syn-en}{Statement}
\newtheorem{remark-syn-ru}{Замечание}
\newtheorem{remark-syn-en}{Remark}
\newtheorem{definition-syn-ru}{Определение}
\newtheorem{definition-syn-en}{Definition}
\newtheorem{example-syn-ru}{Пример}
\newtheorem{example-syn-en}{Example}
%\newtheorem{problem}{Задача}

%\newcommand\norm[1]{\left\lVert#1\right\rVert}
%\newcommand\normx[1]{\left\Vert#1\right\Vert}
\providecommand{\abs}[1]{\lvert#1\rvert}
\providecommand{\norm}[1]{\lVert#1\rVert}

\def\R{\mathbb{R}}
\def\C{\mathbb{C}}
\def\N{\mathbb{N}}
\def\Z{\mathbb{Z}}
\def\Q{\mathbb{Q}}

\def\L{\mathcal{L}}	% norm L_x
\def\cal#1{\mathcal{#1}} % class menoir don't support \cal.
%\renewcommand\qedsymbol{QED}
%\renewcommand\qedsymbol{$\blacksquare$}
%\def\qed{\hfill$\blacksquare$}
%\def\boxw{\hfill$\Box$}

\newcommand{\initSynopsisRU}{
	\counterwithout{figure}{chapter}
	\counterwithout{table}{chapter}
	\renewcommand{\figurename}{Рисунок}
	\renewcommand{\tablename}{Таблица}
	\renewcommand\thesubfigure{\asbuk{subfigure}}
}

\newcommand{\initSynopsisEN}{
	\renewcommand{\thesubfigure}{\alph{subfigure}}
	\renewcommand{\figurename}{Figure}
	\renewcommand{\tablename}{Table}
	\setcounter{figure}{0}
	\setcounter{table}{0}
	\setcounter{equation}{0}	% Reset equation numbering https://tex.stackexchange.com/questions/207532/reset-equation-numbering-after-each-section
}

\newcommand{\intChapter}{
	\counterwithin{figure}{chapter} % Hình ảnh theo chapter, e.g Figure 1.1
	\counterwithout{table}{chapter} % Table thì ko cần, độc lập.
}

\newcommand{\initAppendix}{
	\appendix
	% Оформление заголовков приложений ближе к ГОСТ:
	\setlength{\midchapskip}{20pt}
	\renewcommand*{\afterchapternum}{\par\nobreak\vskip \midchapskip}
	%\renewcommand\thechapter{\Asbuk{chapter}} % Чтобы приложения русскими буквами нумеровались
}
